\section{Interview 1 : Cultured Meat Industry}

\textbf{Interviewer:} Good morning! Thank you so much for taking the time to speak with me today. I really appreciate it. So how are you and how is life going .\newline
\textbf{Manager:} Good morning! No problem at all. I understand the painpoint of thesis student so no worries.  Actually, it's quite refreshing to talk about what we're doing here. I am good. Just came from home country after visiting my parents How's your thesis coming along? \newline

\textbf{Interviewer:} (laughs) Well, just finished my literature review section and hope it will be good from now… Yes, it's definitely been an interesting journey. I'm at Maastricht University working on middle management's role in sustainable innovation, specifically comparing your industry with clean energy. \newline
\textbf{Manager:}  Ah, Maastricht! We work closely with the university. Great choice of topic too. Good to see our roles are in spotlight now other than innovation in startups. Everyone is like whats new.. whats technology but good to know we are also highlighted. \newline

\textbf{Interviewer:} That's exactly what makes it so fascinating and important to study. How long have you been with Mosa Meat? \newline
\textbf{Manager:} About four years now. Started when we were much smaller - it's been quite the ride watching us grow. We started small but now growing rapidly so that's nice. \newline

\textbf{Interviewer:} so what is mosa meat means to you?\newline
\textbf{Manager:} oh…. Other than my work, it is a purpose driven objective I would say.. it is doing better for the world everyday, better for animals especially cow and hope to continue the same.\newline

\textbf{Interviewer:} that's really thoughtful and good to know. Let's dive into the main topic. Can you tell me about the biggest operational challenges you face, particularly around energy and resource efficiency?\newline
\textbf{Manager:} (leans forward) Oh, well everyday there is some operational challenges. But thinking of few or one of the biggest challenges which we face every day is… (long pause) the intensive energy requirement for running bioreactors at scale. We are required to maintain sterile conditions and precise temperature... which consumes a lot of energy and these bioreactors doesn't stop any minute for operations to maintain cell cultures.\newline

\textbf{Interviewer:}  could you elaborate ?\newline
\textbf{Manager:} I mean .You have to understand - these aren't just simple containers. We're essentially creating perfect little environments for cells to grow, and cells can be quite... finicky, shall we say? (laughs) They need everything just right - temperature, pH, nutrients, oxygen levels. It's like being a very expensive babysitter for billions of tiny creatures. \newline

\textbf{Interviewer:} How do you address these energy challenges?\newline
\textbf{Manager:} for this specific problem, we regularly monitor energy and water usage, set reduction targets, and focus on process innovation. But here's where I get excited - (gestures enthusiastically) - for example, on the growth medium side, we have made huge progress. Since 2020, we reduced the cost of our growth medium by 80 fold and our fat medium by 66 fold. That's not just good for our bottom line, that's good for sustainability too. Lower costs often mean more efficient processes, and more efficient processes typically mean less energy consumption per unit of product.\newline

\textbf{Interviewer:} oh that's impressive! other than operational challenges, what other challenges in your opinion you have faced in this industry?\newline
\textbf{Manager:} there are challenges regarding regulations, funding challenges, scaling challenges.. there are many like managing stakeholder, managing supply chain, keeping up with innovations but depending on different departments.\newline 

\textbf{Interviewer:} That's a lot I guess. How do you navigate the regulatory landscape?\newline
\textbf{Manager:} (sighs) Well, that's... that's an adventure, let me tell you. Regulations regarding novel food is always changing. We are part of many regulatory bodies across Europe, USA, and UK. We are actively engaged with Food Standards Agency in UK. They have a regulatory 'Sandbox' program where we are participating to directly consult with regulators in our development process. It's actually quite collaborative, which is nice. We're not just throwing applications over the fence and hoping for the best. We're working together to understand what safety means in this new context. Sometimes I think I spend more time in regulatory meetings than I do looking at actual meat! (laughs). \newline

\textbf{Interviewer:} you said challenge regarding stakeholder whats that? Is it like engagement and partnerships? \newline
\textbf{Manager:} That's crucial for us. We check the credibility of our partners at all tiers and work with only those with whom we share the vision, but it is very broad to explain. Let me give an example - we collaborated with Nutreco to transform food grade amino acids to minimize costs and environmental impact. \newline
We also achieved B Corp certification last year, which was a big milestone for us. It's not just about the certification itself, but what it represents - that we're serious about using business as a force for good. When you're trying to change the entire food system, you need partners who believe in that mission too. \newline

\textbf{Interviewer:} How do you approach workforce development in such a novel industry? \newline
\textbf{Manager:} (becomes more animated) We have a very diverse team. We have  , scientist, engineers, food technologists, and umm... and other people too - regulatory experts, business development, you name it. We also invest a lot in giving training especially as we are growing. For example, we have around 120 employees from I think more than 20 countries.\newline

The diversity is amazing - you'll hear Dutch, English, German, Spanish all in one meeting. We also have ongoing programs in bioprocessing, quality assurance, and sustainability as well. You know what's funny? Half our team probably couldn't have imagined they'd be working in "meat" when they started their careers!  \newline
Cross-functional collaboration is absolutely critical. Our lab team needs to talk to our regulatory team, who needs to talk to our business team, who needs to talk to our sustainability team. Everyone's expertise matters.\newline

\textbf{Interviewer:} Finally for this section . haha.. that's for me.. I will take more of your time.., I am interested in knowing how do you approach scaling up while maintaining sustainability? \newline
\textbf{Manager:} (pauses) That's... that's the ultimate challenge, isn't it? Scaling up is a major focus.  Honestly, we're not yet at the scale as like other sectors, but we do by stakeholder collaboration and by constantly reviewing our processes and partnerships to ensure that as we scale, we maintain our ..um..sustainability ah. commitments.  \newline

\textbf{Interviewer:} So, um, let's talk a bit about how you and your team have approached, you know, consumer scepticism. I mean, cultured meat is still pretty new for most people. How did you start tackling that? I read many news articles about this industry and it is still a barrier I assume.. \newline
\textbf{Manager:} Yeah, that's uh, honestly one of the biggest hurdles we've faced, and, you know, it's not just about the technology or the science,  It's about how people feel about what they're eating. We realized early on that the words we use matter a lot—like, when we said "lab-grown meat," people would, um, get this look on their face, like, "Am I a science experiment?" So, we actually spent a lot of time, uh, across different departments especially marketing, R\&D, just discussing what language would make people more comfortable. "Although it emphasizes a lot about marketing or branding thing, but we didn't just make the decision; rather it is more derived from research. Some studies showed that the word cultivated meat felt more appealing and natural." \newline

\textbf{Interviewer:} That's interesting. I remember once my room mate was also doing her internship in TNO for lab grown meat and when she used to tell me I was like no way I am going to eat it but then she always said-it's the future .. lol.. it was around 3 years ago I guess but it is indeed a future… So, was it hard to get everyone on board with that kind of messaging? \newline
\textbf{Manager:} Oh, for sure. I mean, you'd be surprised..scientists are, uh, not always the easiest to convince when it comes to branding!..haha.. But we had these regular meetings, like, every week, where we'd get marketing, lab, and even some of the execs together, and we'd go over, you know, what kind of questions people were asking online, what journalists were saying, and just, like, try to make sure everyone was speaking the same language. It's still a work in progress, honestly, but it's made a real difference in how we present ourselves. \newline

\textbf{Interviewer:} How do you actually reach out to consumers? Like, what does that look like in practice or on ground level? \newline
\textbf{Manager:} So, um, we do a bunch of things. We've done public tastings, which are, uh, both stressful and really rewarding. The first time we did one, I was honestly terrified—like, what if people hate it? But it was great, and, actually, after that event, we saw a big spike in website visits and people signing up for updates. We also work with universities like Maastricht, actually..to run consumer studies and get real feedback. And we try to be as transparent as possible, so we put a lot of info on our website, post on social media, and, um, even invite journalists into the lab to see how things are done. It's about building trust, you know? If people feel like you're hiding something, they'll never buy in. \newline

\textbf{Interviewer:} yes your socials are amazing and your company is also quite transparent with news and all.. but I am interested in knowing have you ever had to change your product or process based on what you heard from consumers? \newline
\textbf{Manager:} Oh, absolutely. One big thing was the use of animal serum in the growth medium. Some people were, like, "Wait, isn't that still using animals?" And, yeah, that was a fair point. So, we made it a priority to develop a serum-free, plant-based medium. It wasn't easy, but once we had it, we didn't just, you know, publish a scientific paper and move on—we actually made a point to tell our audience, our partners, even chefs we work with, that, "Hey, we listened, and we changed."  Through our socials and physical meets ..That kind of transparency, I think, really helped build trust. People want to know you're not just in it for the tech, but that you actually care about their concerns. \newline

\textbf{Interviewer:} And internally, how do you make sure everyone's aligned on this kind of stuff? It sounds like a lot of moving parts. \newline
\textbf{Manager:} Yeah, it's a lot. We have regular cross-team meetings, especially when we're rolling out something new or making a shift—like when we switched from saying "lab-grown" to "cultivated." It was my job, and others at my level, to make sure not just our team but also the board and, um, even investors understood why we were making those changes. Sometimes it's tricky, because, you know, people have different backgrounds—some are scientists, some are business folks, some are marketers—so you have to translate things in a way that makes sense to everyone. But, honestly, I think that's one of the strengths of being in a smaller, fast-moving company; you can get everyone in a room and hash things out pretty quickly. \newline

\textbf{Interviewer:} What about partnerships? Do you think working with outside groups has helped with acceptance? \newline
\textbf{Manager:} Definitely. We've partnered with universities, chefs, restaurants, even other food companies. Those collaborations help us reach people we wouldn't otherwise, and, um, having respected partners vouch for us goes a long way. It's not just about having a cool product; it's about showing that you're credible, that you're not just a startup making wild claims. And, yeah, investors, too when reputable organizations back you, it signals to the market that you're serious, which helps with both consumers and regulators. \newline

\textbf{Interviewer:} Looking back, what do you think has been the most important thing for building acceptance? \newline
\textbf{Manager:} Honestly, it's commitment like, not just from the top, but at every level. I would say especially at ground level because it is us who are constantly answering to everyone investors, consumer, like to everyone... You must be willing to listen, adapt, and, um, sometimes admit you got it wrong. If you're not open to feedback, or if you try to push something people aren't ready for, it just doesn't work. So, yeah, I'd say it's about being transparent, responsive, and actually caring about what people think not just saying you do, but showing it in your actions every day. \newline

\textbf{Interviewer:} wow! That mus be very hard..Any final thoughts on managing in such a disruptive industry? \newline
\textbf{Manager:} You know, every day feels different… some days are low but some are exciting.. everyday is different issues.. Its like ,,um..we're writing the playbook. Traditional food companies have decades or centuries of experience to draw from. We're figuring it out in real-time. But that's also what makes it exciting…we get to make our road.. like we shape.. how this industry develops from the ground. \newline

\textbf{Interviewer:} do you sometimes feel to be the part of any conventional industry? How do you feel about this industry?
\textbf{Manager:} honestly  I love being able to tell people I help make meat without animals. It's not often you get to work on something that could genuinely change the world. Even if it does confuse people at dinner parties hahaha.. but overall it is tough but we learn a lot.  \newline

\textbf{Interviewer:} I think this is such a topic I can go on -it fascinates me how you as a manager bringing science into reality.. its commendable. I also believe innovation is one thing but bringing to life is completely next level. I think you and your team is doing amazing job. Thankyou so much again for your time and such a detailed response.. I really appreciate
\textbf{Manager:} indeed! What you said is true and as I said it is good to know someone is writing on our roles. Thankyou for having me. \newline