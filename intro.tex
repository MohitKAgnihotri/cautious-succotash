
\section{Introduction}

Increasing global demand for sustainable development has transformed industrial approaches. Disruptive sectors like clean energy and cultured meat are growing, driving innovation and new business models (\textcite{SeedBlink2025, Deloitte2025}). Disruptive technologies transform existing markets and business models by introducing new value propositions, replacing existing products, services, or manufacturing processes (\textcite{Bower1995}). Among these emerging sectors, cultured meat and clean energy solutions are increasingly recognized for their potential to address critical challenges such as climate change, resource shortages, and food security (\textcite{Wageningen2022, GFI2025future}). For example, global renewable capacity increased by 50\% from 2022 to 2023, with solar PV and wind accounting for a record 96\% of that growth. This highlights solar energy's primary role in decarbonization. Similarly, research on cultured meat shows it could significantly minimize land use, greenhouse gas emissions, antibiotic use, and improving animal welfare compared to conventional meat. This industry is rapidly expanding; with over 170 companies emerging by 2023 and securing billions in funding based on environmental and ethical claims.

\paragraph*{} Despite these promises, most research examines technological and market aspects while neglecting lower and middle management considerations.

\paragraph*{} Literature on solar-energy extensively examines technical innovations like grid integration and energy storage (\textcite{SINSEL20202271}). Similarly, existing research in cultured meat focuses on production processes and regulatory pathways (\textcite{Bryant2020}). While these sectors offer revolutionary environmental and economic benefits, their widespread adoption and scaling are limited by technical, organizational, and systemic barriers that vary across industries. These examples show that promising green industries are often operationally complex, facing challenges ranging from finance and supply chain logistics to consumer education and regulatory compliance.

\paragraph*{} While management commitment is acknowledged as vital for sustainability in disruptive industries, comparative and sector specific research is lacking to examine how on middle management navigates unique barriers and leverage enablers across different disruptive contexts (\textcite{Lozano2015, Egri2000}).

\paragraph*{} Addressing this research gap is crucial for the tangible benefits these comparative insights can offer. This study of middle management in the disruptive fields of clean energy and cultured meat offers substantial theoretical and practical contributions. 
Theoretically, by examining middle management's role in these novel disruptive contexts, this research seeks to extend theories of sustainable leadership, which have predominantly focused on executive roles, and offer a more granular understanding of how strategy is operationalized in environments characterized by high uncertainty and rapid innovation.
Practically, comparative insights aim to provide actionable strategies for middle managers to more effectively navigate common and sector specific barriers related to technology development, market acceptance, and resource allocation, thereby potentially accelerating sustainability transitions in their respective fields. These insights may also inform policy design and investment strategies aimed at supporting these vital industries.
A nuanced understanding of how middle managers enact sustainability in practice can help organizations in these rapidly evolving sectors to better harness their innovative capacity, overcome critical adoption hurdles, and ultimately enhance their contribution to global sustainability goals such as climate change mitigation and food security.

\paragraph*{} Furthermore, the comparative element in this research is central to achieving these contributions. By placing the developing cultured meat sector with the evolving clean energy industry, this study aims to pinpoint how middle managers drive sustainability. It identifies both industry specific strategies and universal best practices, contributing to developing more robust managerial frameworks for sustainable innovation.
