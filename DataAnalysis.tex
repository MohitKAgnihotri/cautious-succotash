
\section{Data Analysis}

\subsection*{Common Themes}

\subsection{RQ1: How do management strategies differ in addressing challenges in clean energy versus cultured meat?}
This section presents the five key themes that emerged from interviews with middle management in the clean energy and cultured meat sectors. The analysis highlights commonalities and sector-specific differences in how managers approach sustainability challenges.

\subsubsection{Theme 1: Resource and Energy Efficiency}
Resource and energy efficiency are core operational challenges across both sectors, though each addresses them in different ways due to the nature of their technologies and processes.

\paragraph{Cultured Meat}
Managers in cultured meat companies stress the intensive energy requirements for running bioreactors at scale. Maintaining sterile conditions and precise temperature is energy-intensive, with bioreactors operating continuously to support cell cultures. As one manager explained:
\begin{quote}
	“One of the biggest challenges which we face every day is the intensive energy requirement for running bioreactors at scale. We are required to maintain sterile conditions and precise temperature… which consumes a lot of energy and these bioreactors run 24/7 in operations to maintain cell cultures.”
\end{quote}
To address this, managers regularly monitor energy and water usage, set reduction targets, and focus on process innovation. A significant area of progress was in growth media formulation. One manager provided an example of this progress:
\begin{quote}
	“For example, on the growth medium side, we have made huge progress like since 2020, we reduced the cost of our growth medium by 80 fold and our fat medium by 66 fold.”
\end{quote}

\paragraph{Clean Energy}
For clean energy firms, energy efficiency is both an objective and an operational necessity. Managers focused on efficiency through sustainable product design and resource use. Materials were selected not only for performance but also for recyclability and environmental impact.
\begin{quote}
	“We use non-toxic silicon, and we have 99\% recyclable plastic which reduces the need for aluminum or glass. So yes, efficiency I would say is the biggest sustainability challenge.”
\end{quote}
Managers also described ongoing efforts to optimize production processes to minimize waste and carbon footprint:
\begin{quote}
	“The solar foil we have uses less than half of the energy than conventional solar panels, so I think we are always working on reducing waste.”
\end{quote}

\subsubsection{Theme 2: Regulatory Adaptation and Compliance}
While both sectors face regulatory pressures, the approaches differ due to the maturity and novelty of the technologies involved.

\paragraph{Cultured Meat}
Regulatory adaptation is a dynamic and ongoing process. Managers described close engagement with regulatory agencies and participating in pilot programs as a key strategy to ensure compliance with evolving standards.
\begin{quote}
	“Regulations regarding novel food is always changing. We are part of many regulatory bodies across Europe, USA, and UK. We are actively engaged with Food Standards Agency in UK. They have regulatory ‘Sandbox’ program where we are participating to directly consult with regulators in our development process.”
\end{quote}
The need for proactive regulatory management was underscored by the novelty of the sector and lack of harmonized standards.

\paragraph{Clean Energy}
In contrast, clean energy companies operate in a more stable regulatory environment. The managers focused on adherence to established certifications and standards, often through partnerships.
\begin{quote}
	“We do have partners like TU Delft through which we adhere to compliance. We also have IEC certification for our products. But I cannot recall where we had to adapt on the go because these are market regulations which keep changing everyday…”
\end{quote}
While regulatory change is still a challenge, the sector’s greater maturity means compliance is often managed through established channels rather than constant adaptation.

\subsubsection{Theme 3: Stakeholder and Supply Chain Engagement}
Effective stakeholder management was critical in both sectors, but the types of stakeholders and engagement strategies varied.

\paragraph{Cultured Meat}
Stakeholder engagement is multifaceted and requires collaboration across the value chain, i.e., involving collaboration with suppliers, universities, and investors. Managers highlighted the importance of shared vision and credibility.
\begin{quote}
	“We check the credibility of our partners at all tiers and work with only those with whom we share the vision, but it is very broad to explain. Let me give an example, we collaborated with Nutreco to transform food-grade amino acids to minimize costs and environmental impact.”
\end{quote}
Managers also highlighted social and community impacts such as B Corp certification and partnership with local suppliers.

\paragraph{Clean Energy}
Clean energy managers identified investors and local suppliers as key stakeholders. Interviews with managers emphasized the role of partnerships in supporting market expansion.
\begin{quote}
	“Stakeholders, investors, even our collaborations are very important in our industry. We constantly work with all the stakeholders for scaling up or market reach.. like I am not sure if you have heard of Vopak, they are our major investor along with others and they help us in market reach. For our suppliers, we try to work with local suppliers, and we have our standard process to check their credibility.”
\end{quote}

\subsubsection{Theme 4: Workforce Transformation and Skills Development}
\paragraph{Cultured Meat}
Managers described significant investment in workforce development, with a focus on diversity and ongoing training. With rapid growth, ensuring skill alignment was seen as vital to scale operations.
\begin{quote}
	“We have a very diverse team. We have scientists, engineers, food technologists, and umm.. and other people. We also invest a lot in giving training especially as we are growing. For example, we have around 120 employees from I think more than 20 countries…. We also have ongoing programs in bioprocessing, quality assurance, and sustainability as well.”
\end{quote}
Cross-functional collaboration and a culture of innovation were emphasized as critical for scaling from R\&D to commercial production.

\paragraph{Clean Energy}
While less detailed in the responses, clean energy managers also noted the importance of technical upskilling to support expansion and maintain quality standards.

\subsubsection{Theme 5: Scaling Up Sustainable Manufacturing}
\paragraph{Clean Energy}
Scaling production was identified as a core challenge, particularly balancing growth with sustainability.
\begin{quote}
	“Currently we are a small production facility operating in less than 40MWp here in Arnhem, but we do have plans to scale up to 300MWp. But this scaling up is not that easy, like we must maintain quality, work on minimizing waste.”
\end{quote}
Managers described dedicated teams for managing end-to-end sustainability, and regular progress monitoring was established to manage this balance.

\paragraph{Cultured Meat}
While not as detailed, managers did mention the importance of process management and stakeholder collaboration to ensure sustainability during scale-up.

\begin{table}[h!]
	\centering
	\caption{Summary of Managerial Approaches to Operational Challenges (RQ1)}
	\label{tab:rq1_summary}
	\begin{tabularx}{\textwidth}{@{}lXX@{}}
		\toprule
		\textbf{Theme} & \textbf{Cultured Meat Approach} & \textbf{Clean Energy Approach} \\
		\midrule
		Resource \& Energy Efficiency & Bioreactor optimization, growth media cost reduction & Material innovation, recyclable design, process lean \\
		\addlinespace
		Regulatory Adaptation & Active engagement, regulatory sandboxes, evolving standards & Established certifications, compliance partnerships \\
		\addlinespace
		Stakeholder/Supply Chain & Collaboration with suppliers, universities, B Corp, investors & Investor relations, local suppliers, standard checks \\
		\addlinespace
		Workforce/Skills & Diverse teams, ongoing training, cross-functional culture & Technical upskilling, quality management \\
		\addlinespace
		Scaling Up & Process management, stakeholder collaboration & Dedicated teams, end-to-end sustainability, monitoring \\
		\bottomrule
	\end{tabularx}
\end{table}

\subsection{RQ2: What role does middle management commitment play in mitigating consumer scepticism or market acceptance in disruptive industries?}
This section identifies the major themes that emerged from interviews with middle management in the cultured meat and clean energy sectors regarding their role in addressing consumer scepticism and facilitating market acceptance. The analysis highlights both shared and sector-specific strategies.

\subsubsection{Theme 1: Strategic Language and Framing}
\paragraph{Cultured Meat}
In the interview, managers emphasized reframing the product to reduce psychological barriers. By shifting specific terms like “lab-grown” or “cultured” to “cultivated meat,” managers coordinated efforts across departments—from marketing to labs—to ensure consistent, appealing messaging which adheres to scientific reasoning.
\begin{quote}
	“We realised early that language in novel food matters the most. Different managers across departments worked together to communicate and shift the terminology from “lab-grown” or “cultured meat” to “cultivated meat.””
\end{quote}
\begin{quote}
	“Although it emphasizes a lot about marketing or branding thing, but we didn’t just make the decision; rather it is more derived from research. Some studies showed that the word cultivated meat felt more appealing and natural.”
\end{quote}
This strategic reframing enhanced public perception and broader industry adoption of the term, leading to more market acceptability than before.

\paragraph{Clean Energy}
Managers in this sector emphasized building credibility through independent certifications and high-profile partnerships, allowing them to reach a wider market and gain broader visibility and credibility.
\begin{quote}
	“We focused on building credibility by acquiring some independent certificates like ATEX Zone I. Apart from this, I think partnership with some recognized companies like Vopak and Groendus gave us visibility and consumer trust maybe.. we showcased our technology in demanding environments, and it helped us in gaining consumer’s trust.”
\end{quote}

\subsubsection{Theme 2: Education, Transparency, and Public Engagement}
\paragraph{Cultured Meat}
Educating people and being transparent with them were the core of building trust. Managers organised public tastings, partnered with universities for consumer studies, and used direct communication channels.
\begin{quote}
	“We believe in educating people about novel food because if something is unknown to people, they would never have acceptance. For this reason, we make our partnerships with universities such as yours, Maastricht University, for consumer study. We learned the importance of being transparent as accurate and reliable information is key to acceptance. Other than this, we also host public tastings, work with chefs……… we use our company’s website and social media to explain processes and idea.”
\end{quote}
Managers also responded directly to specific consumer concerns, such as the use of serum, by prioritizing and communicating scientific breakthroughs.
\begin{quote}
	“We were using animal serum which wasn’t well accepted among some consumers, so we tried the development of serum-free, plant-based medium. Fortunately, we succeeded and then we decided to share it with our audience and not just in scientific society…”
\end{quote}

\paragraph{Clean Energy}
Clean energy managers also focused on transparency and education, but they opted for a different route, i.e., through technical documentation, patents, FAQs, and live demonstrations.
\begin{quote}
	“We think education through documents is very important in this sector because of its high-tech nature. So, we have detailed technical documentation, etc…. we also share experiences of customers who share their story, and it gains the trust of future clients.”
\end{quote}

\subsubsection{Theme 3: Responsive Product and Process Innovation}
\paragraph{Cultured Meat}
Managers actively incorporated consumer feedback into product development, such as serum-free media in response to ethical concerns.
\begin{quote}
	“We prioritized the development of a serum-free, plant-based medium. This transparency helped us to reassure the customer’s trust in us regarding safety and ethics.”
\end{quote}

\paragraph{Clean Energy}
Feedback from sceptical customers drove improvements in product documentation and the installation process.
\begin{quote}
	“We are constantly adopting from the feedback of early adopters and pilot project partners. We even use feedback to improve user documentation and installation guides to make it easier for people. In research also we adapt technology based on feedback.”
\end{quote}

\subsubsection{Theme 4: Internal Alignment and Cross-Functional Coordination}
\paragraph{Cultured Meat}
Middle management ensured that all teams were aligned on messaging and prepared to address public concerns, using regular cross-functional meetings.
\begin{quote}
	“We regularly have cross-team meetings to make sure that different departments understand each other. It is still an early stage, so these meetings are important and not that difficult to manage. We make sure that marketing and the lab team are on the same team. For example, when we made a shift from lab-grown meat to cultivated meat, it was our responsibility to explain to the board and top managers so they can further explain to investors and other stakeholders.”
\end{quote}

\paragraph{Clean Energy}
Managers coordinated between R\&D, production, sales, and customer support to ensure the whole organisation responded effectively to market feedback.
\begin{quote}
	“Communication among different departments is important. We try to ensure that R\&D and production are on the same page, and marketing and sales as well.”
\end{quote}

\subsubsection{Theme 5: Leveraging External Partnerships and Investor Credibility}
External partnership and investor backing were used to build credibility and accelerate market acceptance.
\begin{quote}
	“Our partnership with investors like Invest-NL, and others have been very crucial. They not only provided the capital but also laid a foundation for the market where reputable organisations believed in us.”
\end{quote}
While less prominent in the cultured meat data, the sector also gained from strategic collaboration with research institutions, universities, etc. to build trust.

\begin{table}[h!]
	\centering
	\caption{Summary of Managerial Approaches to Market Acceptance (RQ2)}
	\label{tab:rq2_summary}
	\begin{tabularx}{\textwidth}{@{}lXX@{}}
		\toprule
		\textbf{Theme} & \textbf{Cultured Meat Approach} & \textbf{Clean Energy Approach} \\
		\midrule
		Strategic Language \& Framing & Terminology shift (“cultivated”), sector-wide messaging alignment & Certifications, pilot partnerships, credibility building \\
		\addlinespace
		Education \& Transparency & Public tastings, consumer studies, open communication & Technical docs, demos, FAQs, site visits \\
		\addlinespace
		Responsive Innovation & Serum-free media, addressing ethical concerns & Product improvements, user-focused documentation \\
		\addlinespace
		Internal Alignment & Cross-functional meetings, unified messaging & Regular feedback loops, cross-departmental coordination \\
		\addlinespace
		External Partnerships & Academic collaborations, industry alignment & Investor credibility, pilot projects with major partners \\
		\bottomrule
	\end{tabularx}
\end{table}

\subsection{Cross-Sector/Emergent Themes: Lessons for Middle Management}
Drawing on the patterns and findings from both RQ1 and RQ2, several cross-sector themes emerge that highlight how middle management in both cultured meat and clean energy sectors drive sustainable innovation and market acceptance. These themes offer actionable lessons that transcend sector boundaries and can inform best practices for middle management in other emerging industries.

\subsubsection{Proactive Framing and Strategic Communication}
\noindent\textbf{Lesson:} Middle management in both sectors recognized that how innovations are communicated to the public and stakeholders is just as important as what is being presented. They strategically reframe language (e.g., “cultivated” vs “lab-grown” meat) and emphasized credible partnerships (e.g., ATEX certification) to build legitimacy and lower the psychological barriers to adoption.

\subsubsection{Education, Transparency, and Public Engagement}
\noindent\textbf{Lesson:} Both sectors understand that transparency and educating the public or making them aware are crucial to overcoming scepticism. Middle management took efforts in providing clear and accessible information, organizing public demonstrations or tastings, and directly addressing consumer concerns.

\subsubsection{Responsive Innovation Driven by Stakeholder Feedback}
\noindent\textbf{Lesson:} Middle management in both sectors actively sought and responded to stakeholder feedback, using it to guide product and process innovation. This responsiveness built credibility and demonstrated a commitment to continuous improvement.

\subsubsection{Internal Alignment and Cross-Functional Coordination}
\noindent\textbf{Lesson:} Achieving internal alignment across departments was essential for consistent messaging, rapid adaptation, and effective problem-solving. Middle management played an important role in organising cross-functional teams and regular feedback loops.

\subsubsection{Building Credibility Through External Partnerships}
\noindent\textbf{Lesson:} Partnerships with credible organisations—varying from investors, industry peers, or research institutions—were a catalyst in accelerating market acceptance and signalling authenticity to sceptical stakeholders.

\subsubsection{Adaptive Leadership in Uncertain, Evolving Contexts}
\noindent\textbf{Lesson:} Both sectors demand middle management to have an adaptive leadership style due to rapidly changing regulatory and market environments, balancing compliance, innovation, and stakeholder engagement while being flexible to new obstacles and opportunities.

\begin{table}[h!]
	\centering
	\caption{Summary of Cross-Sector Themes and Examples}
	\label{tab:cross_sector_summary}
	\begin{tabularx}{\textwidth}{@{}lXX@{}}
		\toprule
		\textbf{Cross Sector Theme} & \textbf{Cultured Meat} & \textbf{Clean Energy} \\
		\midrule
		Strategic Communication \& Framing & Shift to “cultivated meat,” unified messaging & Certifications, pilot partnerships \\
		\addlinespace
		Education \& Transparency & Public tastings, open science communication & Site visits, technical documentation \\
		\addlinespace
		Responsive Innovation & Serum-free media in response to concerns & Product/process improvements via feedback \\
		\addlinespace
		Internal Alignment & Cross-functional meetings, message training & Cross-departmental coordination \\
		\addlinespace
		External Partnerships & University/industry collaborations & Investor/public partnerships \\
		\addlinespace
		Adaptive Leadership & Navigating new regulations, regulatory pilots & Adapting to evolving standards and markets \\
		\bottomrule
	\end{tabularx}
\end{table}
