\section{Interview 2 : Clean Energy Industry}
\textbf{Interviewer:} Thanks for taking the time to speak with me today. How are you doing? \newline
\textbf{Manager:} Sure. No worries. I am good. Pretty much caught up with the work. How are you? \newline

\textbf{Interviewer:} Good Good. I would just get started with my questions looking at your time availability. I'd like to start by asking, what are the main challenges you face in your role at HyET Solar?
\textbf{Manager:} Sure. Well, I think, um, the biggest challenge is really about efficiency across the board. You know, solar is already seen as clean energy, but if you look at the full production cycle, there's still a lot to optimize. For us, it's not just about producing panels or foils that generate electricity, um.. but ..about how we make them, what materials we use, and so on,  and how much energy goes into the process. You know things like this.. but we use non-toxic silicon, and, uh, our plastic is 99\% recyclable, which helps reduce the need for aluminium or glass. So, efficiency is, I'd say, the biggest sustainability challenge. \newline

\textbf{Interviewer:} Can you give an example of how you try to improve efficiency in your operations? \newline
\textbf{Manager:} Yeah, so, for instance, our solar foil uses less than half the energy compared to conventional solar panels. That's something we're proud of. We're always working on reducing waste, too. I mean, every step in the production line is monitored. if we see excess material loss or higher than expected energy use, we review and adjust. Sometimes that means tweaking the roll to roll deposition process, sometimes it's about better training for the operators, it is different always.. like not entirely but kinda.. \newline

\textbf{Interviewer:} and what about regulations? I read that it is always evolving...How do you keep up with regulations in such a fast-evolving sector? \newline
\textbf{Manager:} Regulations are yes indeed changing but I'd say, a constant but manageable part of the job. The solar sector is a bit more mature than some other clean tech fields, so there are established standards. We do have partners like TU Delft to make sure we're always compliant. We have IEC certification for our products, and, honestly, most changes are incremental—like, updates to testing protocols or safety checks. It's not often we have to scramble to adapt, but we do keep an eye on regulatory updates. \newline

\textbf{Interviewer:} What about working with external partners or investors..how does that factor into your day-to-day? \newline
\textbf{Manager:} That's actually quite central. We have our investors, for example, Vopak, aren't just financial backers they're also pilot customers. We test our foil on their storage tanks, which gives us real-world feedback. That helps us improve the product and, at the same time, builds credibility in the market. For suppliers, we try to work with local companies whenever possible. We have a standard process for checking their credibility, and we look for partners who can keep up with our sustainability standards. \newline

\textbf{Interviewer:} And internally, how do you make sure your team has the right skills to keep up with all these changes?
\textbf{Manager:} Technical upskilling is important for us, especially as we expand. We do a lot of in-house training, and we also run workshops with TU Delft. The technology is evolving, so we need people who are comfortable learning new processes like, for roll-to-roll manufacturing or thin-film deposition. It's not always easy to find those skills in the market, so we invest in developing them internally. \newline

\textbf{Interviewer:} Scaling up production is a big topic in clean energy. How are you managing that at HyET? \newline
\textbf{Manager:} That's definitely a challenge. Right now, we're a relatively small facility.. less than 40MWp I think but we have plans to scale up to 300MWp. Scaling up isn't just about adding more machines. We need to maintain quality, minimize waste, and make sure our sustainability targets aren't lost in the process. We have dedicated teams for monitoring energy use, waste, and supply chain impacts as we grow. It's a balancing act, honestly. \newline

\textbf{Interviewer:} What do you see as the biggest difference between how your sector and, say, something like cultured meat, approaches these challenges? \newline
\textbf{Manager:} Well, I'd say, um, the clean energy sector solar in particular is a bit further along in terms of regulation and public acceptance. There's a clearer framework for what's expected, and the technology is, you know, proven. Our focus is on making things better, more efficient, and more scalable. I think in newer sectors, managers might spend more time justifying the technology or navigating uncertainty. For us, the challenge is to keep improving within a system that's already established. \newline

\textbf{Interviewer:} That's helpful. Is there anything else you'd like to add about your role or the company's approach to sustainability? \newline
\textbf{Manager:} I guess I'd just say that, uh, for us, sustainability isn't a separate project..it's built into how we work. Every decision, from materials to partnerships to training, is measured against that standard. It's not always easy, but it's what sets us apart. And, you know, it's what keeps us moving forward. \newline

\textbf{Interviewer:} Thanks for explaining your production and scaling strategies. Now I am in my second RQ (nothing changes for you) but I'd like to shift gears a bit and ask about how you and your team deal with market acceptance and, you know, consumer or client scepticism. How do you see your role as a manager in that process? \newline
\textbf{Manager:} Yeah, um, so, I think, in our sector, trust is really everything. Even though solar is, you know, well established compared to some other clean technologies, there's still a lot of hesitation especially when you're introducing something new, like our flexible solar foils. People want to know if it's reliable, if it's safe, if it'll last. 

\textbf{Interviewer:} How do you actually go about building that credibility? Is it mostly about technical data, or do you focus on partnerships as well? \newline
\textbf{Manager:} It's definitely both. We put a lot of effort into getting independent certifications like, for example, ATEX Zone I, which is a pretty demanding safety standard. Having that certificate, it's not just a box-ticking thing; it's something we can show to customers and say, "Look, this isn't just our word this has been verified by a third party." And, yeah, partnerships are crucial. Working with companies like Vopak or Groendus, who are recognized in the industry, gives us a lot of visibility. When clients see our technology in use at those companies, especially in tough environments, it really helps break down initial scepticism. \newline

\textbf{Interviewer:} what do you think is keeping people away from using the technology like why not world wide adoption at rapid scale..  \newline
\textbf{Manager:} umm nice one! I think people aren't very aware or have full awareness or education in this area.  \newline

\textbf{Interviewer:} so you are saying that education or transparency plays a role in how people perceive your product? \newline
\textbf{Manager:} Absolutely. I mean, a lot of our customers are technically minded, so we make sure to provide detailed technical documentation …we think education through documents is very important in this sector because of its high-tech nature, so we have detailed technical documentations  like..installation guides, mostly FAQs, even patent information up to some extent. We also share case studies from existing customers. It's not just about saying, "Here's what we do," but actually showing how it works in practice. And, um, we're open about the challenges too. If there's a limitation or something we're still improving, we don't hide it. I think that honesty actually builds more trust in the long run. \newline

\textbf{Interviewer:} that's actually nice way to educate people.. because mostly companies focus on selling but making them understand is a good step ...so Have you ever had to adjust your approach based on feedback from clients or early adopters?
\textbf{Manager:} Yeah, definitely. After some of our first pilot projects, we got feedback that some of our installation instructions were, uh, a bit too technical or not clear enough for everyone. So, we revised the documentation, added more visuals, and even did some hands-on training sessions. We also use feedback to improve the product itself..like, if a customer points out a design issue or a practical challenge, we'll bring that back to the R\&D team and see if we can address it in the next iteration.

\textbf{Interviewer:} How do you make sure everyone internally is aligned on this approach..like, R\&D, sales, customer support? \newline
\textbf{Manager:} That's a good question and a tough in practice. But communication is key. We have regular meetings where we bring together people from different departments…R\&D, production, marketing, sales. The idea is to make sure everyone understands what the customer is experiencing, what questions keep coming up, and where we might need to adapt our messaging or our product. It's not always easy, but I think it's essential for consistency and, honestly, for building confidence both inside and outside the company. \newline

\textbf{Interviewer:} And what about your investors or external partners do they play a role in market acceptance? \newline
\textbf{Manager:} For sure. They actually act as mentors sometimes.. Having investors like Invest NL on board, or partnerships with established companies, really helps. It's not just about the funding; it's about the signal it sends to the market. If reputable organizations believe in us, it reassures potential customers that we're not just a startup with a nice idea. we're a serious player. Sometimes, those partners also open doors to new markets or pilot opportunities, which is invaluable for building a track record. \newline

\textbf{Interviewer:} Looking back, what would you say is the most important thing a manager can do to help new technology gain acceptance? \newline
\textbf{Manager:} I think it comes down to commitment and follow-through. You have to be willing to listen, to adapt, and to keep the lines of communication open with your team, with your partners, and with your customers. It's about being transparent, responsive, and consistent. If you do that, you build trust, and trust is what ultimately moves the needle on acceptance. \newline

\textbf{Interviewer:} That's pretty much it! I think I got everything .. thankyou so much for your time and valuable insughts..and if I need would it be okay to ask you through mail or linkedIn? \newline
\textbf{Manager:} no worries. That would be fine. Thankyou to you too and have a nice rest of the day.  \newline
