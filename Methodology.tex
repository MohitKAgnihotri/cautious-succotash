\section{Methodology}
This study employed a qualitative, comparative research design to explore the role of middle management in driving sustainable innovation within the disruptive contexts of clean energy and cultured meat. The approach was exploratory, aiming to understand the unique barriers, enablers, and strategies that shape managerial practices in these two rapidly evolving sectors.

\subsection{Participant Selection and Recruitment}
A purposeful sampling strategy was used to identify and recruit participants. The criteria for inclusion required participants to be in middle management roles with direct responsibilities for operational, innovation, or sustainability-related functions within their organizations. This targeted approach ensured that all participants possessed relevant experience and could provide informed perspectives on the research questions.

The final sample consisted of six middle managers, with three participants from the clean energy sector and three from the cultured meat industry. This comparative structure was designed to draw out both shared and industry-specific insights.

\subsection{Data Collection}
Data were collected through semi-structured, in-depth interviews conducted virtually. This format allowed for a consistent line of inquiry guided by the research questions while also providing the flexibility to explore emergent topics raised by the participants. An interview guide was developed based on the key themes identified in the literature review, focusing on managerial strategies, operational challenges, and stakeholder engagement.

Each interview lasted approximately 60 minutes. With permission from the participants, all interviews were audio-recorded and transcribed verbatim to ensure the accuracy of the data for analysis.

\subsection{Data Analysis}
The interview transcripts were analyzed using a thematic analysis approach to identify, analyze, and report patterns within the data. The analysis was conducted in several phases, beginning with a thorough familiarization with the transcripts to gain a comprehensive understanding of the participants' experiences.

Following this, the data were systematically coded to identify key concepts and recurring ideas. These codes were then organized into potential themes. The analysis involved a constant comparison between the data from the clean energy and cultured meat sectors, which served to highlight both common strategies and sector-specific distinctions. This comparative process was central to addressing the research questions and generating nuanced findings. Finally, the themes were reviewed and refined to ensure they accurately represented the data.

\subsection{Trustworthiness and Ethical Considerations}
To ensure the trustworthiness of the findings, several measures were taken. The use of a semi-structured interview guide enhanced the consistency of data collection, while the detailed, verbatim transcripts provided a reliable foundation for analysis.

Ethical considerations were paramount throughout the study. All participants were informed of the study's purpose and procedures and provided informed consent before their interview. To protect confidentiality, all identifying information, including the names of participants and their organizations, was anonymized during transcription and analysis.