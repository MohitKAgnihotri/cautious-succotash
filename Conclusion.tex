
	
	\section{Discussion \& Conclusion}
	
	Our qualitative analysis reveals that middle management in both the cultured meat and clean energy industries see themselves as crucial players in driving sustainability forward, but they put into practice this role in notably different ways across the two sectors.
	
	\paragraph*{} In the clean energy sector, which is characterized by more established technologies and regulatory structures, managers primarily described their efforts as focused on making gradual improvements, ensuring compliance with existing regulations, and boosting efficiency. This approach aligns with the well-known concept highlighted by \citeauthor{Fischhoff2024}, where meeting compliance often signifies success (\textcite{Fischhoff2024}). For instance, several project managers in renewable energy stated that their main environmental objective was to meet or surpass government standards. Their sustainability initiatives largely revolved around optimizing processes, such as enhancing plant efficiency or integrating existing green technologies.
	
	\paragraph*{} Conversely, the emerging cultured meat industry pushed its managers to adopt a more entrepreneurial and innovation-driven stance. Cultured meat managers emphasized building new relationships with regulators, research institutions, and investors. They also focused on educating stakeholders and developing positive perceptions of the technology. Their goal was to establish the legitimacy of a relatively new product, rather than simply adhering to regulations.
	
	\paragraph*{} Despite these differences between the sectors, a common thread emerged: managers in both contexts performed what is known as the "middle-up-down" function. This involves translating high-level sustainability strategies into practical operational plans and, in turn, relaying feedback from operations back up to senior leadership. This bridging function is particularly important for embedding sustainability within an organization, as \citeauthor{Posch2017} identify middle managers' "middle-up-down role" in performance measurement as key to embedding sustainability (\textcite{Posch2017}). Our findings indeed showed that our participants described aligning team incentives and performance metrics with corporate sustainability goals, demonstrating the virtues-based staffing and cultural alignment mechanisms suggested by \citeauthor{Posch2017}.
	
	\paragraph*{} Our analysis also revealed that managers in both industries identified similar factors driving their sustainability efforts, consistent with \citeauthor{Lozano2015}'s comprehensive model of sustainability drivers. He proposes a holistic framework that categorizes these drivers into internal and external factors, emphasizing their inter-connectedness in fostering corporate sustainability (\textcite{Lozano2015}).
	
	\paragraph*{} From an internal perspective, our interviewed managers frequently highlighted leadership commitment and the direct business case as primary motivators. For example, visible support for "green" initiatives from executives and clear benefits related to cost savings or enhanced reputation were key drivers for action. This aligns with \citeauthor{Lozano2015}'s research, which identifies "leadership and the business case" as among the most significant internal drivers for integrating sustainability into corporate strategy.
	
	\paragraph*{} Externally, factors such as pressure from regulations, customer demand, and brand image were consistently mentioned. This too directly mirrors \citeauthor{Lozano2015}'s finding that "reputation, customer demands, and regulation" are crucial external drivers for companies to adopt sustainable practices.
	
	\paragraph*{} For instance, managers in the clean energy sector often referred to regulatory incentives and market differentiation, such as "green branding," as significant external drivers, echoing \citeauthor{Lozano2015}'s external factors. Similarly, cultured meat managers, despite operating in a newer field, also perceived market demand and consumer attitudes as powerful external forces. Many spoke about leveraging public interest in sustainability to attract investment or gain market entry. In both sectors, a clear business case whether it was cost savings from efficiency in clean energy or projected economies of scale in cultured meat provided internal justification for managers to pursue sustainability initiatives, reinforcing the idea of dual internal and external drivers found in existing literature.
	
	\subsection{Sectoral Interpretations and Literature Comparison}
	Our observations regarding the distinct approaches in clean energy and cultured meat align with and expand upon previous research concerning how industry context influences sustainability leadership. For instance, \citeauthor{Fischhoff2024}'s interviews with energy managers highlighted a predominant focus on regulatory compliance rather than on initiatives that went beyond it (\textcite{Fischhoff2024}). Our findings among clean energy managers echoed this sentiment: most cited adherence to regulations as their primary sustainability endeavor, with very few mentioning innovations that exceeded these requirements. This suggests that \citeauthor{Fischhoff2024}'s conclusion remains relevant in contemporary renewable energy contexts.
	
	\paragraph*{} In stark contrast, managers in the cultured meat industry, operating with minimal existing regulations, reported actively engaging in proactive dialogue with policymakers and pioneering new industry norms. This distinction also connects with research on consumer acceptance of novel foods. \citeauthor{Bryant2020} indicate that addressing cultural and emotional barriers such as concerns about naturalness, disgust, or ethical implications is vital for the widespread adoption of cultured meat (\textcite{Bryant2020}). Our study found that cultured meat managers were indeed engaged in public outreach and educational efforts, acutely aware of these issues. One interviewee, for example, described developing messaging campaigns that highlighted animal welfare and health benefits rather than focusing solely on technical novelty. This approach is supported by \citeauthor{Bryant2020}'s observation that purely technical explanations often "failed to persuade" skeptical consumers. Essentially, cultured meat managers are implementing the literature's recommendation to emphasize personal and ethical advantages to overcome stakeholder resistance.
	
	\paragraph*{} When compared to established theories of sustainability leadership, our findings particularly highlight the significant role played by middle management. \citeauthor{Eccles2014} illustrate how, in companies with robust sustainability cultures, boards and executives integrate environmental, social, and governance (ESG) objectives into their governance structures, often linking sustainability metrics to incentives (\textcite{Eccles2014}). We observed complementary evidence at the managerial level: in many firms, middle management served as crucial links, translating these formal governance commitments into practical actions. For example, in several companies identified as having strong sustainability practices, our participants noted that they were explicitly evaluated based on environmental Key Performance Indicators (KPIs). This reflects the "long-term orientation" and "sustainability metrics" characteristics that Eccles et al. describe. Thus, our results broaden \citeauthor{Eccles2014}'s top management perspective by demonstrating how a sustainability culture is enacted "in the trenches" by middle managers. Similarly, Winston, \citeauthor{Winston2023} argue that middle management is the "unsung heroes" in the corporate sustainability journey (\textcite{Winston2023}). Our study confirms their assertion: even organizations with progressive commitments often face operational limitations, making empowered middle management critical for driving actual change. For instance, managers reported challenges like resource shortages and departmental silos that hindered sustainability progress precisely the kind of implementation gaps that \citeauthor{Winston2023} note can cause well-intentioned strategies to "fall short."
	
	\subsection{Theoretical Contributions}
	This research contributes significantly to both sustainability leadership theory and disruption management by highlighting the interplay between industry context and managerial agency.
	
	\paragraph*{} First, we refine sustainability leadership theory by shifting the focus to the middle management level, providing a more granular understanding of how strategic sustainability goals are translated into operational practices under conditions of high uncertainty and rapid innovation. While the existing literature predominantly examines senior managers, CEOs, and boards of directors who set strategic direction and initiate sustainability initiatives, our qualitative data demonstrate that middle managers also exercise leadership by embedding values and by aligning day-to-day operations with long-term ESG goals. They serve as crucial intermediaries, bridging strategic directives and daily operations, and translating strategic sustainability goals into practical actions, often managing emergent challenges during implementation. This context-driven understanding enriches theories of distributed or responsible leadership.
	
	\paragraph*{} Second, by comparing an upcoming "disruptive" sector (cultured meat) with a more established one (renewable), we develop comparative frameworks that distinguish management approaches in emerging versus maturing disruptive industries. Our findings suggest that sustainable innovation necessitates different managerial behaviours depending on the sector's maturity and operational complexity. In the cultivated meat industry, characterized by high uncertainty and nascent institutional structures, middle managers behave more like entrepreneurs focused on foundational R\&D innovation and managing consumer perceptions, establishing initial market footholds, overcoming fundamental production hurdles, and addressing consumer skepticism about novel food technologies. In clean energy, by contrast, managers engaged more in tasks related to scaling technologies and navigating policy or infrastructure barriers, optimizing existing processes, managing large-scale infrastructure projects, and navigating established yet complex stakeholder environments. This conditional insight linking the nature of disruption to managerial orientation adds a contextual layer to both sustainability and disruption theory by underscoring the importance of context-specific middle management approaches in disruptive environments. In summary, we extend existing models by identifying industry-specific leadership patterns: for example, in cultured meat, the key managerial challenge is building legitimacy for an unproven product, whereas in energy it is optimizing within existing regulatory regimes and ensuring compliance. These nuanced contributions help bridge sustainability leadership with strategic management perspectives on innovation and industrial context.
	
	\subsection{Practical Implications}
	This research offers clear practical implications stemming from the observed differences between the sectors.
	
	\paragraph*{} For company leadership, our findings underscore the critical importance of empowering and adequately training middle management. Organizations should formally acknowledge middle management's pivotal role as a conduit for sustainability implementation. This can be achieved, for example, by integrating sustainability targets directly into middle management's performance goals and by involving them in strategic planning processes. The recommendations by \citeauthor{Posch2017} regarding virtues-based staffing and formal alignment can be operationalized through hiring managers who inherently possess strong sustainability values or by providing targeted ethics and Corporate Social Responsibility (CSR) training.
	
	\paragraph*{} In clean energy firms, where our interviewed managers predominantly focused on regulatory compliance, executives should consider expanding their mandate. Middle managers could be tasked with piloting initiatives that go beyond mere regulation to stimulate innovation, such as testing new renewable technologies or exploring novel demand response programs. Conversely, in cultured meat companies, managers require support in their stakeholder engagement roles. Firms might establish cross-functional teams where mid-level R\&D, marketing, and policy managers collaborate specifically on public outreach efforts. We also recommend that cultured meat organizations dedicate resources to communication strategies that highlight personal and societal benefits of their products, as existing literature suggests this framing resonates more effectively with consumers.
	
	\paragraph*{} For policymakers and industry organizations, our results emphasize how the role of middle managers can be effectively leveraged. In the cultured meat sector, public sector entities and industry consortia could involve middle managers in regulatory pilot programs. This would enable these managers to directly contribute to shaping safety standards and supply-chain norms. Furthermore, training workshops or innovation grants could be specifically targeted at mid-level managers to accelerate technology scaling and market introduction. In the clean energy sector, regulators might engage energy companies' project managers in advisory groups, ensuring that policy developments are well-aligned with on-the-ground operational realities.
	
	\paragraph*{} More broadly, both sectors would benefit significantly from knowledge-sharing forums where middle management can exchange best practices. Examples include workshops focused on embedding sustainability metrics into project management, thereby supporting their crucial "middle-up-down" role, or on effective strategies for communicating sustainability initiatives both internally and externally. Organizations could also formalize the role of "sustainability champions" at the mid-management level, establishing designated liaisons between strategic teams and operational units. By taking these steps, firms can translate our theoretical insights into concrete organizational designs that foster a culture where sustainability is not solely a top-down directive but is actively co-created with middle managers, aligning with \citeauthor{Eccles2014}'s emphasis on robust stakeholder engagement procedures.
	
	\subsection{Limitations and Future Research}
	This study, while offering valuable insights, is subject to several limitations that concurrently open promising avenues for further research.
	\begin{enumerate}
		\item Our data are derived from a focused qualitative sample across just two industries. Although interviews provide rich, in-depth understanding, they inherently cannot capture the full spectrum of experiences within middle management. Therefore, future investigations could enhance generalizability by employing larger-scale surveys or comparative case studies that encompass diverse firm sizes, varied geographic regions, and specific sub-sectors (e.g., distinct types of renewable energy technologies or different meat substitutes).
		
		\item The cross-sectional nature of our design presents a snapshot in time. Given that the cultured meat sector, in particular, is rapidly evolving, the roles and priorities of managers within it are likely to undergo significant shifts as the industry matures. Longitudinal research would be invaluable here, allowing researchers to track middle managers' strategies over an extended period to observe how their focus transitions for instance, from early-stage R\&D coordination to market launch tactics.
		
		\item Our analysis concentrated solely on the perspectives of part of the management team. A more holistic understanding of multi-level organizational dynamics could be achieved by complementing these insights with the views of senior executives and frontline employees. For example, future studies might explore how middle managers' sustainability initiatives translate into measurable outcomes at various levels or delve into the intricacies of how they negotiate objectives and resources with both top leadership and operational teams.
		
		\item While we highlighted significant differences between the cultured meat and clean energy sectors, other critical contextual factors were not systematically examined. These include national policy environments, nuanced consumer cultural attitudes, or varying firm ownership structures. Conducting cross-cultural comparisons for instance, by studying European versus Asian contexts could reveal how local societal and regulatory frameworks modify sustainability leadership practices.
	\end{enumerate}
	
	In conclusion, we strongly advocate for continued exploration of middle management's role in driving sustainability, particularly within emerging industries. Subsequent studies could aim to develop and empirically test models that quantify how middle management influences firm performance across environmental, social, and governance (ESG) dimensions, or how they navigate the inherent tensions between conflicting goals such as profit generation and broader purpose-driven initiatives. By addressing these identified limitations, future research can contribute to building a more comprehensive and nuanced theory of how sustainability leadership operates effectively at all organizational levels and across a diverse array of sectors.
	