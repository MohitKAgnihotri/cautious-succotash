\section*{3. Methodology}

This study employed a qualitative, comparative research design to conduct an in-depth exploration of the role of middle management in driving sustainable innovation within disruptive industries. A qualitative approach was selected as it is uniquely suited for understanding the complex processes, strategies, and contextual factors that shape managerial practices—insights that are not easily captured through quantitative methods. The research design is fundamentally exploratory, aiming to generate nuanced understanding of the barriers and enablers in the rapidly evolving sectors of clean energy and cultured meat.

\subsection*{3.1 Research Design and Rationale}

The research utilizes a comparative design, focusing on two distinct disruptive sustainability sectors: clean energy and cultured meat. This comparative approach is strategic, as the differing maturity levels of these industries provide a valuable lens for analysis. 
\begin{itemize}
	\item The \textbf{clean energy} sector represents a relatively mature yet continuously evolving industry, where managers often focus on scaling existing technologies and navigating established policy frameworks.
	\item The \textbf{cultured meat} sector exemplifies an emerging industry, where managers must overcome fundamental R\&D hurdles and establish initial market legitimacy in a context of consumer uncertainty and nascent regulation.
\end{itemize}
By comparing these two cases, this study can distinguish between universal managerial challenges and context-specific strategies, thereby enriching theoretical frameworks for sustainable innovation.

\subsection*{3.2 Participant Selection and Recruitment}

Participants were selected using a \textbf{purposeful sampling strategy} to ensure all individuals possessed deep and relevant knowledge of the subject matter. 
\begin{itemize}
	\item \textbf{Inclusion Criteria:} The primary criterion for inclusion was that participants must hold a middle management position with direct operational, innovation, or sustainability-related responsibilities within their organization. This targeted approach guaranteed that participants could offer informed perspectives directly related to the research questions.
	\item \textbf{Sample Composition:} The final sample comprised six middle managers, with three participants from the clean energy sector and three from the cultured meat industry. This balanced, comparative structure was intentionally designed to facilitate the identification of both shared and industry-specific insights.
	\item \textbf{Recruitment Process:} Participants were identified and recruited through professional networks. Given this targeted sampling method, a formal response rate was not calculated. To protect participant confidentiality, specific demographic data such as age or gender were not collected.
\end{itemize}

\subsection*{3.3 Data Collection}

The primary method for data collection was \textbf{semi-structured, in-depth interviews}. This format was deliberately chosen for its dual strengths: it ensures a consistent line of inquiry across all interviews while providing the flexibility necessary to probe emergent topics and participant-specific insights.
\begin{itemize}
	\item \textbf{Interview Instrument:} An interview guide was developed based on the key themes and research gaps identified in the literature review. The questions were designed to elicit detailed responses regarding managerial strategies, operational challenges, resource allocation, and stakeholder engagement in relation to the study's research questions.
	\item \textbf{Procedure:} Each interview was conducted virtually and lasted approximately 60 minutes. With explicit permission from each participant, all interviews were audio-recorded to ensure the complete and accurate capture of data, which were subsequently transcribed verbatim for analysis.
\end{itemize}

\subsection*{3.4 Data Analysis}

A \textbf{thematic analysis} approach was employed to systematically analyze the verbatim transcripts. This method is highly effective for identifying, analyzing, and reporting patterns (or themes) within rich qualitative data. The analysis was conducted as a rigorous, multi-phase process:
\begin{enumerate}
	\item \textbf{Familiarization:} The process began with a thorough review of all transcripts to achieve a deep and holistic understanding of the participants' narratives.
	\item \textbf{Systematic Coding:} Following familiarization, the data were systematically coded to identify key concepts, ideas, and recurring patterns relevant to the research questions.
	\item \textbf{Theme Generation:} The generated codes were then collated and organized into potential overarching themes. A process of \textbf{constant comparison} between the data from the two sectors was central to this phase, serving to highlight both common strategies and sector-specific distinctions.
	\item \textbf{Review and Refinement:} Finally, the identified themes were reviewed and refined to ensure they accurately and coherently represented the dataset as a whole, directly addressing the core research questions.
\end{enumerate}

\subsection*{3.5 Trustworthiness and Ethical Considerations}

\subsubsection*{Trustworthiness}
To ensure the trustworthiness of the findings, several measures were implemented. The use of a semi-structured interview guide enhanced the consistency and dependability of data collection. The practice of creating detailed, verbatim transcripts provided a credible and reliable foundation for the thematic analysis.

\subsubsection*{Ethical Considerations}
Ethical conduct was paramount throughout the study's lifecycle. All participants were fully informed about the research purpose and procedures before providing their informed consent to participate. To guarantee confidentiality and protect participants, all identifying information, including the names of individuals and their organizations, was fully anonymized during the transcription and data analysis stages.